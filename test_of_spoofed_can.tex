Testing of the spoofed GNS was split intro two subtests. \\
The first test was conducted in order to validate the GPS spoofing using the CAN-bus works when the drone is laying on a table. \\
Test 1 can be seen in \ref{sec:test_of_spoofed_positions_indoore}.\\
The second test was conducted as the first test, but when the drone is airborn.\\
Test 2 can be seen in \ref{sec:test_of_spoofed_positions_outdoore}



----------------------------------\\
Figure Y shows the waypoint list uploaded to the drone. It is expected that the drone will behave as if it was using its onboard GPS.
The GPS positions will be gathered in a rosbag to be used later on in test2.
A flight with the onboard GPS will be done in order to have a reference flight.\\
\textbf{Testx}
Test two will be conducted much the same way as test1. However the GPS will be replaced with a lightweight RTK GPS. The RTK GPS positions will also be saved in a rosbag for later analyse. Is is expected that when using the RTK GPS the drone is closer to its waypoints  shown in figure Y, than when using a normal GPS.

----------------------------------\\





\textit{This section describes the overall system architecture indoor as well as outdoor. The section goes in details with the components used and how they are connected. Furthermore it describes the information flow through the system. The design choices and implementation details will be covered later in this chapter.}

The section has been split into two subsections in order to address indoor and outdoor positioning separately.

\subsection{Indoor} \label{sec:system_architecture_indoor}
Figure \ref{fig:indoor_information_flow} shows the system architecture for indoor positioning. The laptop detects the drones position using vision, sends it through the ROS nodes and transmits the GPS positions to the drones via WIFI.
Each drone receives its position and sends it into the M4 flight controller over \ac{CAN}.
\begin{figure}[H]
    \center
    \includegraphics[width=1\textwidth]{graphics/system_architecture_indoor_all.png}
    \caption{Diagram showing system architecture when used indoor. The markers used to track the drones is shown below the camera.}
    \label{fig:indoor_information_flow}
\end{figure} 

A Logiteh C930 camera\footnote{The camera is used in other courses to track mobile robots so the author did not have any influence on which camera to use} was mounted below the ceiling to detect the drones when flying. The order of the marker of each drone is unique in order to identify the particular position of the drone.


\subsubsection*{ROS}
\ac{ROS}\footnote{http://www.ros.org/} is used as middleware on the laptop as inter-process communication. By using ROS it is easier to debug since each node\footnote{Process in \ac{ROS} terminology} can be isolated and debugged without everything has to be connected. \ac{ROS} uses subscriber-publisher pattern which means one or more nodes can produce data, and one or more nodes can consume data. Topic is the term used to define a communication channel between nodes. Nodes producing data are referred to as publishers, and nodes consuming data are referred to as subscribers.
Furthermore \ac{ROS} supports running nodes distributed across multiple PCs meaning the MarkerLocators can be distributed if more CPU resources are needed when tracking several drones. The main focus when designing the system was, that it is scalable so that the design of the system is not causing a limitation of the number of drones that can be controlled by the system.
The squares shown in the \ac{ROS} section in figure \ref{fig:indoor_information_flow} shows the \ac{ROS} nodes and how they are connected using topics.


\begin{itemize}
	\item \textit{cam\_to\_topic} captures frames from the Logitech camera and publish the raw frame to \textit{/camera/down} as a RGB picture.
	\item \textit{MarkerLocator\_trackN} subscribes to \textit{cam\_to\_topic} and processes the frame in order to localize the marker or order N of the drone in the frame. This node publishes messages to \textit{/positions/droneN} containing x,y position, orientation and a boolean telling whether the right marker is found in the frame or not. The MarkerLocator was customized to support receiving frames from a topic instead of the camera directly. By doing this, multiple instances of the MarkerLocator can be run in parallel without their performance impact each other. \footnote{Indoor localization is described in section \ref{sec:indoor_localization}}
	\item \textit{Decision\_Maker} subscribes to the topics containing the position of the drones. This node publishes messages to \textit{/communication/to\_droneN} containing the \ac{LLH} position of the drone. This node contains the logic for controlling the drones \footnote{Decision\_maker is described in section \ref{sec:control_and_coordinate_conversion}}. Due to the modularity of ROS, changing the behavior of the drones is a matter of replacing this node.
	\item \textit{wifi\_outN} subscribes to \textit{/communication/to\_droneN} and transmits the received messages to the drone using WIFI. The responsibility of this node is to pack the message, calculate \ac{CRC}, append it and send it as an \ac{UDP} packet. The content of the frame can be seen in table \ref{tab:wifi_frame}.
\end{itemize}

\begin{table}[H]
\centering
\begin{tabular}{@{}|l|l|l|l|l|l|l|l|l|l|@{}}
\toprule
\textit{Content}  & CRC-16    & Future  & eDOP  & nDOP & tDOP & vDOP &      Height   & Lon     & Lat       \\ \midrule
\textit{Datatype} & uint16\_t & 4 byte  & byte  & byte & byte & byte &      double   & double  & double  \\ \midrule
\textit{Bytes}    & 33:32     &  31:28  & 27    & 26   & 25   & 24   &      23:16    & 15:8    & 7:0       \\ \bottomrule
\end{tabular}
\caption{Table shows the frame sent from the \textit{wifi\_out\_N} to the extension-boards. 4 bytes is not utilized but can be used for anything or the frame can be reduced in size}
\label{tab:wifi_frame}
\end{table}

The frame in table \ref{tab:wifi_frame} is 34 bytes long but can be enlarged or made smaller if needed. However the ESP8266 module has a limitation of 8192 bytes \footnote{\url{https://github.com/esp8266/Arduino/blob/master/libraries/ESP8266WiFi/src/WiFiUdp.h\#L28} last visited 28 Maj}

The initial idea was to also send the velocities of the drone and the accuracies to each extension-board since this information is used by AQ M4, however due to lack of time this was not implemented.

\subsubsection*{Drone} \label{sec:system_indoor_drone}
The drone is a Ladybird using a \ac{AQ} M4 as flight controller with a modified version of the firmware.\footnote{AQ M4 firmware is described in section \ref{sec:aq_m4_firmware}} Is has an extension-board that enables it to receive positions from WIFI and inject them into the \ac{AQ} M4 over \ac{CAN}. \footnote{The extension-board is described in section \ref{sec:autoquad_extension_board}}
\begin{itemize}
	\item \textit{ESP8266}\footnote{The firmware running on the ESP8266 module is described in section \ref{sec:exp8266_firmware}} receives the UDP packet sent from the computer running \ac{ROS}. When a frame is received, it checks if the received number of bytes it correct. If so, it encapsulates the packet using \ac{SLIP} and transmit it to the At90CAN128.
	\item \textit{At90CAN128} receives the encapsulated packet, decapsulate it, verify the CRC and transmit the packet to the \ac{AQ} M4 over \ac{CAN}. Tables \cref{tab:CAN_DOC_LAT,tab:CAN_DOC_LON,tab:CAN_DOC_VEL,tab:CAN_DOC_ALT,tab:CAN_DOC_DOP,tab:CAN_DOC_ACC} show the CAN-messages send to the \ac{AQ} M4.
\end{itemize}


\begin{multicols}{2}
\begin{table}[H]
	\begin{tabular}{@{}|l|l|@{}}
		\toprule
		DOC   & CAN\_DOC\_LAT \\ \midrule
		Value & Latitude      \\ \midrule
		Bits  & 63:0          \\ \bottomrule
	\end{tabular}
	\caption{CAN message to \ac{AQ} containing 8 byte latitude}	\label{tab:CAN_DOC_LAT}

	\begin{tabular}{@{}|l|l|@{}}
	\toprule
		DOC   & CAN\_DOC\_LON \\ \midrule
		Value & Longitude     \\ \midrule
		Bits  & 63:0          \\ \bottomrule
	\end{tabular}
	\caption{CAN message to \ac{AQ} containing 8 byte longitude} 	\label{tab:CAN_DOC_LON}

\end{table}
\columnbreak
\begin{table}[H]
	\begin{tabular}{@{}|l|l|l|l|l|@{}}
		\toprule
		DOC   & \multicolumn{4}{l|}{CAN\_DOC\_VEL} \\ \midrule
		Value & VelN    & VelE   & VelD   & speed  \\ \midrule
		Bits  & 63:48   & 47:32  & 31:16  & 15:0   \\ \bottomrule
	\end{tabular}
	\caption{CAN message to \ac{AQ} containing velocities each of 2 bytes}
	\label{tab:CAN_DOC_VEL}
	\begin{tabular}{@{}|l|l|@{}}
	\toprule
		DOC   & CAN\_DOC\_ALT \\ \midrule
		Value & Altitude      \\ \midrule
		Bits  & 63:0          \\ \bottomrule
	\end{tabular}
	\caption{CAN message to \ac{AQ} containing 8 byte altitude}
	\label{tab:CAN_DOC_ALT}
\end{table}
\end{multicols}

\begin{table}[H]
	\begin{tabular}{@{}|l|l|l|l|l|l|l|l|@{}}
		\toprule
		DOC   & \multicolumn{7}{l|}{CAN\_DOC\_DOP}                  \\ \midrule
		Value & gDOP  & eDOP  & nDOP  & tDOP  & vDOP  & hDOP & pDOP \\ \midrule
		Bits  & 55:48 & 47:40 & 39:32 & 31:24 & 23:16 & 15:8 & 7:0  \\ \bottomrule
	\end{tabular}
	\caption{CAN messages to \ac{AQ} containing \ac{DOP}s each of 1 byte}
	\label{tab:CAN_DOC_DOP}
	\begin{tabular}{@{}|l|l|l|l|l|l|l|l|@{}}
		\toprule
		DOC   & \multicolumn{7}{l|}{CAN\_DOC\_ACC}                          \\ \midrule
		Value & Heading & vAcc  & hAcc  & cAcc  & sAcc  & Fix  & Satellites \\ \midrule
		Bits  & 63:48   & 47:40 & 39:32 & 31:24 & 23:16 & 15:8 & 7:0        \\ \bottomrule
	\end{tabular}\label{tab:CAN_DOC_ACC}
	\caption{CAN messages to \ac{AQ} containing accuracies of 1 byte and heading in 2 bytes} 
\end{table}

\subsubsection*{Network configuration}  \label{sec:network_configuration}
The WIFI  network for handling the communication between the PC and the drones had to be configured \footnote{Wireless communication chosen in  section \ref{sec:wireless_communication_module}}.
The PC was configured as an access point for each extension-board to join. The access point was setup using an Asus USB-N13\footnote{Not available on AUSUS' webpage.} netcard and hostapd\footnote{https://w1.fi/hostapd/} as software. The network was configured as \ac{WPA2} to make it difficult for other to connect and start messing with the network.

Instead of hardcoding an  \ac{IP} for each extension-board, which would be cumbersome each time a new firmware had to be deployed, a \ac{DHCP} server was configured on the PC. Since the IP might change next time the module connects, Multicast-DNS was configured on the ESP8266 and the PC. Each extension-board was giving a hostname eg. Drone1 so when the PC has to send a frame to Drone1, the underlaying networking will resolve the IP of Drone1.
The ESP8266 has a small filesystem build-in which support creating a configuration file that could contain the hostname.
When flashing the ESP8266 module the configuration file\footnote{Due to lack of time the configuration file was not made, and the hostname was hardcoded in the ESP8266 firmware.} would not be overwritten and thereby still have the hostname of the module. Instead of using hostnames the \ac{IP} of the drone could be entered in the configuration file, however this would make it less portable since the PC would always have to be on the same \ac{IP}-range. It was chosen to use \ac{UDP} packets to transfer data from the PC to the ESp8266 module. \ac{TCP} could have been used since it retransmits packets if they are lost and takes care of integrity check. If using \ac{TCP} and a packet is lost during transmission it will be sent again. However the packet will be outdated when the ESP8266 modules receives the packet. Instead \ac{UDP} was chosen since the data is sent with 5 hz, nothing happens if a packet is lost.


\subsection{Outdoor}
\begin{figure}[H]
    \center
    \includegraphics[width=0.7\textwidth]{graphics/system_architecture_outdoor.png}
    \caption{Diagram shows information flow when used outdoor. The RTK-GNSS provides a absolute position which is read and converted to CAN messages by the Raspberry PI}
    \label{fig:outdoor_information_flow}
\end{figure}

The \ac{RPi} is also running \ac{ROS} since it uses two components from Frobomind\footnote{\url {http://frobomind.org/} last visited 29 Maj}. \begin{itemize}
	\item \textit{serial\_nmea\_node} is used to read and write from the serial port. Further more it decodes the NMEA string sent by the \ac{RTK-GNSS}
	\item \textit{can\_socket\_node} is used to communicate with the \ac{CAN}-adapter\footnote{PEAK-CAN Adapter has been used in this project}
\end{itemize}

When a GPGGA/GPRMC nmea \footnote{\url{http://www.gpsinformation.org/dale/nmea.htm} last visited 28 Maj} message is received from the \ac{RTK-GNSS} its read by the \textit{serial\_nmea\_node} node. It is then sent to the \textit{AQ\_msg\_conv} node which simply converts the content of the GPGGA/GPRMC to CAN-messages shown in \cref{tab:CAN_DOC_LAT,tab:CAN_DOC_LON,tab:CAN_DOC_VEL,tab:CAN_DOC_ALT,tab:CAN_DOC_DOP,tab:CAN_DOC_ACC}  . \\

Even though the positioning is shown using vision and \ac{RTK-GNSS}, other sources of positioning can be used as well. The same code is running on the \ac{AQ} M4 whether vision or \ac{RTK-GNSS} is used, so if other positioning systems should be used one should implement the \ac{CAN} protocol shown in appendix \ref{app:aq_can_protocol}.
If using \ac{CAN} is not possible a \ac{RPi} and \ac{CAN}-adapter can be used as intermediate, but then the \textit{serial\_nmea\_node} and \textit{AQ\_msg\_conv} should be replaced.


It can be concluded that it is possible to design a indoor and outdoor positioning system using the same \ac{AQ} firmware. The system was designed to be scalable so that more drones can be controlled by the system. By using ROS the system is modular designed so that the behavior of the drones can be changed without the need of replacing or editing other working components of the system. It requires 6 CAN-messages in order to send the required values to \ac{AQ}.
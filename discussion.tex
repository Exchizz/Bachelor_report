It is possible to design a indoor and outdoor positioning system using the same \ac{AQ} firmware.
The system was designed to be scalable so that more drones can be controlled by the system.
By using ROS on the PC the system is modular designed so that the behavior of the drones can be changed without the need of replacing or editing other working components of the system. A \textit{Decision\_Maker} nodes was made to handle coordinate conversion and control the drones. By moving a 4 order marker along a rectangle in the flight-plane the size of the rectangle can be estimated with an error of 1.2 and 0.9 in with and height respectively. 
The ESP8266 wireless module was chosen based on requirements such as weight, size, documentation available and price. The ESP8266 got the highest score and was mainly chosen since it is a widely used module on the internet. An extension-board was created to act as a bridge between the PC and the drone with an At90CAN128 microcontroller. The At90CAN128 was chosen based on its build-in CAN-controller, its availability, and because of previous experience. Three tests was conducted to test the performance of the ESP8266 wireless module. If the distance is larger than 46 meters the tests shows the latency begins to increase and the CRC packets will arrive without error but will be delayed. If the distance is larger than 85 meters the packets suffers from high latency and errors starts to occur.
By selecting a \ac{RTCS} scheduler for the At90CAN128 it was possible to run a task every second with a mean and standard deviation of 1.0089 and 0.0042 sec. respectively. By creating a task-diagram it was possible to design a modular firmware for the At90CAN128 and to use queues as communication between the tasks. Furthermore a small firmware without a scheduler for the EPS8266 module were written.
Several tests were made in order to verify the CAN-GNSS injection worked. First by doing tests indoor and then moving the testing outside.
The outdoor tests shows it is possible to inject CAN-GNSS positions into \ac{AQ} and that \ac{AQ} can be told how much its UKF should trust the GNSS positions based on the DOP values.
When using the MarkerLocator with the improved order detection it is possible to detect two drones without having false positive/negative. By making a test of the distance between two markers there was an error of 0.81 cm.




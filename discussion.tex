It is possible to design an indoor and outdoor positioning system using the same firmware on the \ac{AQ} M4 board.
Since ROS was chosen the nodes can be split into several PCs if more resources is required when more drones are added to the system.
The \textit{Decision\_Maker} node designed to handle coordinate conversion and control the drones was tested by moving a marker along a rectangle. The size of the rectangle could be estimated with an error of 1.2 cm and 0.9 cm in width and height respectively. These error estimates does however contain the uncertainty created by moving the marker by hand and the uncertainty by making the homography. When the drones are flying, a higher precision is expected. I might have been more generic if the \textit{Decision\_Maker} was split into three nodes, instead of doing controlling and 
conversion in the same node. One node to convert the drones position into \ac{UTM}, a node to be the controlling/navigation and a node to convert the UTM-position of the controlling-node into \ac{LL}. 

The ESP8266 wireless module was chosen based on requirements such as weight, size, documentation available and price.
The ESP8266 got the highest score and was mainly chosen since it is a widely used module on the internet. 
Three tests was conducted to test the performance of the ESP8266 wireless module.
If the distance is larger than 46 meters the tests shows the latency begins to increase and the CRC packets will arrive without error but will be delayed.
If the distance is larger than 85 meters the packets suffers from high latency and errors starts to occur. At a distance of 70 meters packets experienced latency up to 1.4 sec. If a frame containing a drones positions arrives 1.4 sec too late it is invalid since the drone could have moved several meters since the frame was sent. Theses results are however obtained outside with a minimum of disturbance so its performance might decrease when used inside with other networks. It was only tested with two drones, however, unless something unexpected shows up, there is reason to believe it also works with 4 drones connected. For instance, more access points can be configured if that turns out to be a limitation.

Several tests were made in order to verify the CAN-GNSS injection worked. First by doing tests on a table and then moving the testing outside in order to avoid drones falling down due to untested code.
The outdoor tests shows it is possible to inject CAN-GNSS positions into \ac{AQ} and that \ac{AQ} can be told how much its UKF should trust the GNSS positions based on the DOP values.
When using the MarkerLocator with the improved order detection it is possible to detect two drones without having false positive/negative. By making a test of the distance between two markers there was an error of 0.81 cm. One measurements should have been done in order to state the MarkerLocator is accurate enough. Some of the error is caused by inaccuracy when creating the homography. When getting close to the boundary of the image, some of the error will also be caused by distortion in the camera. To avoid that, the camera should be calibrated.
\textit{This test was made together with Kjeld Jensen in order get one step further to spoofing the drones onboard GPS with an RTK GPS. A ublox Neo6-P GPS and an active <ANTENNE> was mounted on a EDUQuad drone to obtain the drone's position. A RPI was used on the EDUQuad to convert GPGGA and GPRMC from the Neo6-p GPS to CAN-mesasagess. The tes was done by walking 20 meters with the EduQuad two times to compare the output of the UKF with the different GPS's.
The values used to spoofed the GPS was found by analyzing a logfile from a regular flight to see the range of the values such as accuracy, DOP etc. When the drone is in the air. The values available from the Neo6-p GPS, was used to spoofe the onboard GPS and values not available was either calculated or hardcoded. Three tests were made due to a few coding errors but onyl the first and the last will be described. This test was also relevant for the indoor flying since the indoor application requires the position of the drone can be spoofed and the accuraciesand DOPs can be set}\\
\Mathias{Image of diagram of test connected hardware}
In order to be able to switch between the onboard GPS and the Neo-6P GPS during the test, a switch on the Spectrum DX9\footnote{\url{http://www.spektrumrc.com/Products/Default.aspx?ProdId=SPMR9900} last visited 21. Maj} was read in AutoQuad's firmware to device which of the two GPS's data should be used. When a GPS update(time, position or velocity) is received from the onboard GPS, the struct shown in code \ref{code:gpsData_Struct} is filled in.


\begin{wrapfigure}{r}{0.5\textwidth}
  \begin{center}
\begin{lstlisting}[language = c++, caption = Quality checks added to discard bad positions, label=code:gpsData_Struct]
typedef struct {
  	...
    double lat;
    double lon;
    // above mean sea level (m)
    float height;   
    // horizontal accuracy est (m)
    float hAcc;     
    // vertical accuracy est (m)
    float vAcc;    
    // north velocity (m/s)
    float velN;     
    // east velocity (m/s)
    float velE;     
    // down velocity (m/s)
    float velD;    
    // ground speed (m/s) 
    float speed;   
    // deg 
    float heading;  
    // speed accuracy est (m/s)
    float sAcc;     
    // course accuracy est (deg)
    float cAcc;   
    
    float pDOP;     // position Dilution of Precision
    float hDOP;
    float vDOP;
    float tDOP;
    float nDOP;
    float eDOP;
    float gDOP;

    //Used in CAN spoof
    uint8_t fix;
    uint8_t satellites;
    int out_cnt;
 
    
} gpsStruct_t;
\end{lstlisting}
  \end{center}
  \end{wrapfigure}


In order to spoof the onboard GPS correctly, the author and his supervisor wanted to overwrite all of the elements in the struct, unleash something else has been stated as comments in the struct.



Table \ref{tab:can_messages} shows the CAN-messages created to send the relevant data to fill in the struct:

\begin{table}[]
\centering
\caption{My caption}
\label{my-label}
\begin{tabular}{@{}|l|l|l|l|l|l|l|l|@{}}
\toprule
DOC   & \multicolumn{7}{l|}{CAN\_DOC\_DOP}                  \\ \midrule
Value & gDOP  & eDOP  & nDOP  & tDOP  & vDOP  & hDOP & pDOP \\ \midrule
Bits  & 55:48 & 47:40 & 39:32 & 31:24 & 23:16 & 15:8 & 7:0  \\ \bottomrule
\end{tabular}
\end{table}

% Please add the following required packages to your document preamble:
% \usepackage{booktabs}
\begin{table}[]
\centering
\caption{My caption}
\label{my-label}
\begin{tabular}{@{}|l|l|@{}}
\toprule
DOC   & CAN\_DOC\_LAT \\ \midrule
Value & Latitude      \\ \midrule
Bits  & 63:0          \\ \bottomrule
\end{tabular}
\end{table}

\begin{table}[]
\centering
\caption{My caption}
\label{my-label}
\begin{tabular}{@{}|l|l|@{}}
\toprule
DOC   & CAN\_DOC\_LON \\ \midrule
Value & Longitude     \\ \midrule
Bits  & 63:0          \\ \bottomrule
\end{tabular}
\end{table}

\begin{table}[]
\centering
\caption{My caption}
\label{my-label}
\begin{tabular}{@{}|l|l|l|l|l|l|l|l|@{}}
\toprule
DOC   & \multicolumn{7}{l|}{CAN\_DOC\_ACC}                          \\ \midrule
Value & Heading & vAcc  & hAcc  & cAcc  & sAcc  & Fix  & Satellites \\ \midrule
Bits  & 63:48   & 47:40 & 39:32 & 31:24 & 23:16 & 15:8 & 7:0        \\ \bottomrule
\end{tabular}
\end{table}

% Please add the following required packages to your document preamble:
% \usepackage{booktabs}
\begin{table}[]
\centering
\caption{My caption}
\label{my-label}
\begin{tabular}{@{}|l|l|l|l|l|@{}}
\toprule
DOC   & \multicolumn{4}{l|}{CAN\_DOC\_VEL} \\ \midrule
Value & VelN    & VelE   & VelD   & speed  \\ \midrule
Bits  & 63:48   & 47:32  & 31:16  & 15:0   \\ \bottomrule
\end{tabular}
\end{table}

% Please add the following required packages to your document preamble:
% \usepackage{booktabs}
\begin{table}[]
\centering
\caption{My caption}
\label{my-label}
\begin{tabular}{@{}|l|l|@{}}
\toprule
DOC   & CAN\_DOC\_ALT \\ \midrule
Value & Altitude      \\ \midrule
Bits  & 63:0          \\ \bottomrule
\end{tabular}
\end{table}
2 bytes are allocated for each velocity. In order to send 2 decimals, each velocity was multiplied by 100 when sending and divided by 100 when received by AutoQuad.
Only one decimal for rest of the values was deemed necessary and was thereby multiplied and divided by 10. \\

Figure \ref{fig:dop_read_flight} shows a flight using onboard GPS where the DOP values where estimated. The values listed in table \ref{tab:DOP} was used and multiplied by the hDOP received from the Neo-6P GPS. 


The test was conducted by walking 20 meters \footnote{Measured by counting footsteps} north, 20 meters south, 20 meters east and 20 meters west. After walking 20 meters a small break of five seconds was held. Figure \ref{fig:position_plot} shows the walked path.
\Mathias{Figure med gpgga plot}

The coordinates plotted in figure \ref{fig:position_plot} was from the first test. Figure \ref{fig:ukf_plot} shows a plot of the position with x-axix as time.
\Mathias{Plot af northing and easting fra ukf}
Figure \ref{fig:ukf_plot} matches with the walked path seen in figure \ref{positon_plot}. The line crossing severel times is the switch on the Spectrum DX9 transmitter. It can be seen how the path repeats itself after the switch has been switched.
\Mathias{plot af pos og switch samt dop så man kan se de ændrer sig ved skift.}
Figure \ref{fig:ukf_dop} shows the same plot but where the vDOP and hDOP has been added. It can be seen when zoomed in, how the DOP values is also changing since the DOPs are higher when switching to the Neo-6p GPS. The DOP values expresses how well the satellites are placed. If they are all located around the same spot, the DOP values will be high since the triangulation becomes less accurate. The DOP values has no unit but is simple a scaling of the variance. \cite{gpsbog}


\Mathias{Billede der viser højden/hastigheden er mere woobly siden DOP er mega lav, den stoler for meget på GPS}


Figur der viser højden samt højdehastigheden

figur der viser vele og veln der ikke virker.

Ny test, samme måde som første test, dog med følgende resultater \footnote{fejlen skyltes forkert blevev hevet ud fra C++ vector samt en variabel var brugt forkerte sted.}

\footcite{kelddueholmmikkellaurentziusannab.o.jensen2015}





The list below shows 3 different types of scheduler that was considered.
\begin{itemize}
	\item Real-time Operating System(RTOS) provides strict timing but at the cost of overhead. A RTOS runs task in a "parallel" environment. Each task runs in a loop and the RTOS scheduler will do context switching when needed. Using a RTOS requires mutexes and semaphores to protect shared resources which increases the complexity and amount of overhead.
	\item Super-Loop provides no timing at all, but process data as fast as possible. It does not do any context switching and does not require mutexes or semaphores and  and thereby takes no overhead.
	\item Run-To-Complete(RTCS) scheduler is a mix of the two other schedulers. It works by waiting for a tick generated by a hardware timer and then starts executing all tasks from the beginning. The order of the task matters if there is dependency between the tasks but also to make sure the task requiring the most precise timing is at the beginning of the list of the tasks. All tasks have to be finished executing before the next tick is giving in order to avoid timing is ruined.
\end{itemize}
The RTCS was chosen since it provides timing without the overhead created when doing context-switching and the need of mutex and semaphores. It further reduces the code-complexity.

The firmware was written in C++ to use the same \ac{AQ} CAN-message generation code running on the \ac{RPi}, on the At90CAN128.
The scheduler has been implemented with a minimum functionality.

It supports a software timer which is useful if a \ac{LED} needs to be toggled every x ticks. To use the timer  \textit{wait(ticks)} has to be called from within a task with the number of ticks to wait.
The task will not be executed until the wait period has passed. Each task has its own state-variable which is parsed as a parameter to the task. By calling \textit{set\_state( state )} from within a task, the state-variable can be updated. This was implemented to avoid the need of static or global variables in order for each task to have its own state-machine.

\begin{lstlisting}[language = Matlab, caption = Pseudo code of the RTCS scheduler implemented, label=code:rtcs_implemented]
def scheduler():
	for(ever):
		while(ticks != 1);
		ticks = 0
		for task in tasks:
			if task.task_state == ST_RUN or task.wait_counter== 0:
				task.task_state = ST_RUN
				current_state = task.state
				task.task_ptr(current_state)
			else: // task_state == ST_WAIT
				task.wait_counter--
\end{lstlisting}

\ref{code:rtcs_implemented} shows pseudo code of the scheduler. When tick is different from zero it loops through an array of task. If the current state of the task is ST\_RUN or the wait\_counter is zero, then invoke the callback which is defined when creating a task. If the task is in ST\_WAIT state simply decrement the counter.

Queues was implemented as circular-buffer to handle communication between the tasks. The queues was designed to be generic in order to contains elements of different size. When a queue is created, the size of each element is specified as a parameter.

\begin{lstlisting}[language = C++, caption = Implementation of queues. Notice the queues are generic in size since the size of the element is given as parameter. When a element is put into the queue\, it is done by multiplying the elements size by the index of the next element in the queue and add that to the beginning of the memory allocated for the queue, label=code:rtcs_queue]
// Create queue of 10 elements of 1 byte (main.cpp)
Queue_Uart0_Rx  = QueueCreate(10, sizeof(uint8_t));

// From QueueSend(&Queue_Uart0_Rx, &ch), put element into the allocated memory.
memcpy ( queue->mem + ( (queue->typesize)*(queue->wr) ), dataIn, queue->typesize );
\end{lstlisting}


It was decided to implement the code in tasks to make a low coupling between the functionalities. This makes it easier to maintain and expand later if needed. The task diagram shown in figure \ref{fig:task_diagram_atmega} shows the tasks and how they do inter-task communication using queues.

A hardware timer was configured to set \textit{ticks = 1} at every 1 ms. This means the scheduler is running through all tasks with a frequency of 1khz.


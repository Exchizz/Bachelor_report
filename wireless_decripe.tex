The following wireless modules where considered and compared.
\begin{itemize}
	\item \textbf{ESP8266}
\end{itemize}

ESP8266 is a generel purpose 32 bit SOC with integrated WIFI 802.11 b/g/n support and buildin TCP/IP stack. It can be setup its own access point or it can connect to an existing wireless network.
It runs at 80MHz and can be flashed with a custom firmware. 
The SOC is sold as modules with different pinouts and features such as extra flash memory \footnote{\url{https://www.olimex.com/Products/IoT/MOD-WIFI-ESP8266/open-source-hardware}} and different antennas.
The chip has been on the marked for about two years and costs approximately 7\$. 
It has been widely used in DIY-projects due to its low price and because it requires a minimum of network knowledge to get up and running.\footnote{\url{http://www.esp8266.com/} - 43.000 posts in forum} When the SOC is shipped, it comes with a preloaded firmware which either accepts AT commands or LUA scripting depending on the version of the module. These simple programming interfaces makes it quick and easy to interface the cheap. \\
This leads to a large community where most of the problems have been found and solved already. Arduino has been ported to ESP8266 which makes it even easier to get it up and running. Their official Arduino GitHub has 2125 commits on their master branch at the time of writing\footnote{\url{https://github.com/esp8266/Arduino}} \\

\begin{itemize}
	\item \textbf{EMW3165}
\end{itemize}
EMW3165 is a SOC much like the ESP8266 supporting 802.11 b/g/n WIFI with buildin TCP/IP stack. As with ESP8266 it supports setting up an access point aswell as connecting to an existing network. It has a Cortex-M4 $\mu$C which run at 100MHz. 
It supports custom firmware and can be aswell be bought as different modules with different pinouts and antennas.
It differentiates itself from the ESP8266 by its higher frequency its 5 volts compatible pins \footnote{\url{https://hackadaycom.files.wordpress.com/2015/07/emw3165.pdf}} which makes it easier to connect other hardware which run 5 volt without the need of a logic level shifter. It has been on the marked for only one year and costs approximately 9\$. Since it is a newer board than ESP8266 it has not been used in the same number of applications and thereby has a smaller community behind\footnote{\url{http://www.emw3165.com/} - 200 posts in forum }. Their most active GitHub has 147 commits on their master branch at the time of writing\footnote{\url{https://github.com/SmartArduino/WiFiMCU}}.

\begin{itemize}
	\item \textbf{nRF51822}
\end{itemize}
nRF51822 is also a SOC, but it is using Bluetooth instead of WIFI. The nRF51822 $\mu$C is implementing BLE which is a power efficient way of sending and receiving data. The chip supports broadcasting which could be used in this project. The $\mu$C can be bought as a standalone component or mounted on modules as the two other $\mu$Cs. Different modules offers different types of antenna connectors or buildin antenna on the PCB. It has not been possible to find an Arduino ported firmware that supports this $\mu$C. To write a firmware for the $\mu$C it has to be done using Nordic Semiconductor's proprietary SDK. 


\begin{itemize}
	\item \textbf{XBee}
\end{itemize}
XBee is a module that implements the Zbee standard. The Xbee modules work as a wireless serial connection. The Xbee modules supports mesh networking which means the modules by themself figure out which module is closest and makes the connection. This idea makes sense in this application since there will be multiple drones and one computer. If one drone gets too far from the PC, it can just connect to one of the other drones closer to the PC.\\
The Xbee solution is ready to use and requires a minimum of programming to get up and running. The modules also support GPIO for digital in and output and analog input.



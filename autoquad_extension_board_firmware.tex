\textit{In this chapter the development of the firmwares running on the extension board is described. Different types of schedulers for the At90CAN128 is discussed and a scheduler is chosen and developed with the required functionality. Furthermore the firmware running on the ESP8266 is described. Several tests are conducted to test if it works}
\subsection{Scheduler}
\textit{This capter concerns only the Atmega.  The ESP module will be descriped in section \ref{sec:esp_firmware}} \\
In order to have a timing on the At90CAN128 a scheduler was chosen and implemented.
The list below shows 3 different types of scheduler that was considered.
\begin{itemize}
	\item Real-time Operating System(RTOS) provides strict timing but at the cost of overhead. A RTOS runs task in a "parallel" environment. Each task runs in a loop and the RTOS scheduler will do context switching when needed. Using a RTOS requires mutexes and semaphores to protect shared resources which increases the complexity and amount of overhead
	\item Super-Loop provides no timing at all, but process data as fast as possible. It does not do any context switching and does not require mutexes or semaphores and  and thereby takes no overhead.
	\item Run-To-Complete(RTCS) scheduler is a mix of the two other schedulers. It works by waiting for a tick generated by a hardware timer and then starts executing all tasks from the beginning. The order of the task matters if there is dependency between the tasks but also to make sure the task requiring the most precise timing is at the beginning of the list of the tasks. All tasks have to be finished executing before the next tick is giving in order to avoid timing is ruined.
\end{itemize}
The RTCS was chosen since it provides timing without the overhead created when doing context-switching and the need of mutex and semaphores. It further reduces the code-complexity.\\

It was decided to implement the code in tasks to make a low coupling between the functionalities. This makes it easier to maintain and expand later of if needed. The task diagram shown in figure \ref{fig:task_diagram_atmega} shows the tasks and how they do intertask communication.


\begin{figure}[H]
    \center
    \includegraphics[width=0.9\textwidth]{graphics/task_diagram.png}
  \label{fig:task_diagram_atmega}
  \caption{Tasks diagram showing overview of the running tasks on the At90Can128}
\end{figure}



\subsubsection*{Test of RTCS timing}
In order to test the timing of the scheduler, a led\_task was written. The task can be seen in code \ref{code:test_scheduler}
\begin{lstlisting}[language = c, caption = RTCS task used in timing test, label=code:test_scheduler]
void is_alive_task(uint8_t my_state){

	// Write to UART0
	UDR0 = my_state+'0';

	switch(my_state){
	case 0:
		INT_LED_ON_GREEN;
		INT_LED_OFF_RED;
		INT_LED_OFF_BLUE;
	    set_state( 1 );
		break;
	case 1:
		INT_LED_OFF_GREEN;
		INT_LED_ON_RED;
		INT_LED_OFF_BLUE;
	    set_state( 2 );
		break;
	case 2:
		INT_LED_OFF_GREEN;
		INT_LED_OFF_RED;
		INT_LED_ON_BLUE;

		// Set next state
	    set_state( 0 );
		break;
	}
	// Wait one second
	wait( 1000 );
}
\end{lstlisting}

The test was done by writing the current state of the task to UART0.\\ A python script were made that measures the time between each character received. The script can be seen in code \ref{code:test_rtcs_python}.
\begin{lstlisting}[language = python, caption = Python code used to measure time between received byte, label=code:test_rtcs_python]
#!/usr/bin/python
import serial
import time

ser = serial.Serial( port='/dev/ttyUSB0', baudrate=57600 )

t = time.time()
while True:
    for char in ser.read(1):
        print time.time() - t, ","
        t = time.time()

ser.close()
\end{lstlisting}
The output of the script were redirected to a file. After receiving 700 bytes the standard deviation and mean was calculated using matlab.
The mean is 1.0089 sec with a standard deviation of 0.0042 sec.

Part of the variance is caused by the inaccuracy of the timing on the PC running the python code. If a more accurate measure was needed, a scope could be attached to the $\mu$C's GPIO. Each time the scheduler enters the task the GPIO should be set high, and when it exists the GPIO should be set low. The scopes at SDU is capable of telling the variance of the off signal. \\
It can be concluded that the scheduler performs well.
%\newpage
%\subsection{Tasks}
%\input{tasks}
%\newpage

\subsection{ESP8266 firmware} \label{sec:exp8266_firmware}
The ESP8266 module was initially flashed with the Arduino \footnote{\url{https://github.com/esp8266/Arduino} last visited 29 Maj} bootloader.
The modules supports \ac{OTA} \footnote{\url{https://github.com/esp8266/Arduino/blob/master/doc/ota\_updates/readme.md} last visited 29 Maj} which means the module can be flashed over WIFI instead of using a \ac{FTDI} cable.

At first, the module had to be programmed using an \ac{FTDI} cable.
The ESP8266 has a programm-pin that needs to be held high or low depending on whether the module should boot from its flash or if its about to be programmed.
This pin was connected to the At90CAN128 in order to set the level of the pin without the need of jumpers on the \ac{PCB}. When the At90CAN128 boots, it pulls the programming-pin high or low depending on a define.
It can thus easily be chosen whether the ESP8266 module should boot or be programmed.
After the Arduino bootloader was flashed, there was no need to further use the \ac{FTDI} cable.

The ESP8266 is run in a super-loop meaning it treats UDP packages as fast as possible. The code in \ref{code:esp8266_code} shows the main part of the code running on the ESP8266 module.

\begin{lstlisting}[language = C++, caption = Snippet from loop() shows how it processes frames. When a UDP packet is available\, its size is compared with the expected size of a fame. It then loops through each byte received and runs the SLIP encapsulation, label=code:esp8266_code]
// loop
int packetSize = UDP.parsePacket();    
if(packetSize == PACKET_SIZE){
   	Serial.write(SLIP_END);
    for(int i = 0; i < PACKET_SIZE; i++){
    		switch(packetBuffer[i]){
       		case SLIP_END:
       	      Serial.write(SLIP_ESC);
       	      Serial.write(SLIP_ESC_END);
       	    break;         
       	    case SLIP_ESC:
       	      Serial.write(SLIP_ESC);
       	      Serial.write(SLIP_ESC_ESC);
       	    break;
       	    default:
       	      Serial.write(packetBuffer[i]);
		}
	}
	Serial.write(SLIP_END);
}
// End loop
\end{lstlisting}

The \ac{SLIP} protocol has been used to encapsulate frames sent by the ESP8266 and received by the At90CAN128 in order to know when a frame is beginning and ending. 
The naive implementation of separating frames would be to use a special character as separator. However this fails if the separator occurs as part of string. Instead \ac{SLIP}\footnote{Tee slip protocol is described in details in \url{https://tools.ietf.org/html/rfc1055} last visited 29 Maj} is used to handle escaping the separator if it occurs in the data.

\textbf{Conclusion} \\
It can be concluded that by selecting a \ac{RTCS} it was possible to run a task every second with a mean and standard deviation of 1.0089 and 0.0042 sec. respectively. By creating a task-diagram it was possible to design a modular firmware for the At90CAN128 and to use queues as communication between the tasks. Furthermore a small firmware without a scheduler for the EPS8266 module were written.



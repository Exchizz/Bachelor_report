It can be concluded that it was possible to design a generic system architecture where locations can be injected over CAN into the AQ M4-board using different sources of localization. Tests were conducted in order to verify it was possible to inject absolute positions into AQ. An outdoor test with an RTK-GNSS used in the scientific paper reveals that it works and there by indicates that it also works with a indoor localization system.
An extension-board was created capable of acting as a bridge between a PC and a drone in order to send absolute positions to the AQ-firmware. The extension-board was able to receive 200 packets without CRC error at 10 hz at a distance of 46 meters without loosing any packets or experiencing any noticeable latency. By using the extension-board the 4 drones should be able to fly at a distance between each other of $\pm$ 10 cm without loosing any positions. The communication was also tested with two extension-boards which indirectly verifies the scalability in ROS and the design of the indoor system architecture. This suggests that it also works with 4 drones.
 The MarkerLocator was used as indoor localization and was modified in order to be able to verify the right order was detected. The MarkerLocator showed an accuracy of 0.81 cm which suggests that keeping a distance between 4 drones of $\pm$ 10 cm is possible using the MarkerLocator.\\

Due to lack of time and because of the time spend on the outdoor RTK-GNSS it was not possible make 3 drones follow a leader. Therefor the hypothesis can not be accepted nor rejected since the subparts made fulfills the requirements and suggests it works with the 4 drones.\\

To make it work the system needs to be put together and verify one drone can be controlled. If so, test with more drones and at least replace the current \textit{Decision\_Maker} with a leader-follower algorithm. In case the drone is not capable of holding its height very accurately a distance sensor should be connected on the extension-board. If more advanced control algorithms should be implemented, it might be less resource demanding if the coordinate conversion is moved into two other node, before and after the \textit{Decision\_Maker}-node.
If controlling multiple drones it will fast be an issue concerning the MarkerLocator. An efficient way to lover the amount of resources required would be to port the MarkerLocator into C++.
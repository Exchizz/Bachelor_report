The ESP8266 module was initially flashed with the Arduino \footnote{\url{https://github.com/esp8266/Arduino} last visited 29 Maj} bootloader.
The modules supports \ac{OTA} \footnote{\url{https://github.com/esp8266/Arduino/blob/master/doc/ota\_updates/readme.md} last visited 29 Maj} which means the module can be flashed over WIFI instead of using a \ac{FTDI} cable.

At first, the module had to be programmed using an \ac{FTDI} cable.
The ESP8266 has a programm-pin that needs to be held high or low depending on whether the module should boot from its flash or if its about to be programmed.
This pin was connected to the At90CAN128 in order to set the level of the pin without the need of jumpers on the \ac{PCB}. When the At90CAN128 boots, it pulls the programming-pin high or low depending on a define.
It can thus easily be chosen whether the ESP8266 module should boot or be programmed.
After the Arduino bootloader was flashed, there was no need to further use the \ac{FTDI} cable.

The ESP8266 is run in a super-loop meaning it treats UDP packages as fast as possible. The code in \ref{code:esp8266_code} shows the main part of the code running on the ESP8266 module.

\begin{lstlisting}[language = C++, caption = Snippet from loop() shows how it processes frames. When a UDP packet is available\, its size is compared with the expected size of a fame. It then loops through each byte received and runs the SLIP encapsulation, label=code:esp8266_code]
// loop
int packetSize = UDP.parsePacket();    
if(packetSize == PACKET_SIZE){
   	Serial.write(SLIP_END);
    for(int i = 0; i < PACKET_SIZE; i++){
    		switch(packetBuffer[i]){
       		case SLIP_END:
       	      Serial.write(SLIP_ESC);
       	      Serial.write(SLIP_ESC_END);
       	    break;         
       	    case SLIP_ESC:
       	      Serial.write(SLIP_ESC);
       	      Serial.write(SLIP_ESC_ESC);
       	    break;
       	    default:
       	      Serial.write(packetBuffer[i]);
		}
	}
	Serial.write(SLIP_END);
}
// End loop
\end{lstlisting}

The \ac{SLIP} \footnote{Inspired by SLIP implementation provided by Kjeld Jensen} protocol has been used to encapsulate frames sent by the ESP8266 and received by the At90CAN128 in order to know when a frame is beginning and ending. 
The naive implementation of separating frames would be to use a special character as separator. However this fails if the separator occurs as part of string. Instead \ac{SLIP}\footnote{The slip protocol is described in details in \url{https://tools.ietf.org/html/rfc1055} last visited 29 Maj} is used to handle escaping the separator if it occurs in the data.
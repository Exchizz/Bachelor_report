\textit{This captor concerns only the At90CAN128.  The ESP8266 module is described in section \ref{sec:exp8266_firmware}} \\
In order to have a timing on the At90CAN128 a scheduler was chosen and implemented.
The implementation is described in appendix \ref{app:scheduler_describe}
\begin{figure}[H]
    \center
    \includegraphics[width=0.9\textwidth]{graphics/task_diagram.png}
  \caption{Tasks diagram showing overview of the running tasks on the At90CAN128}
    \label{fig:task_diagram_atmega}
\end{figure}
A small description of each of the tasks is given below:
\begin{itemize}
\item \textbf{is\_alive\_task} Toggles the green led to make sure the scheduler is running.
\item \textbf{uart0\_tx\_task} Responsible of sending characters in the uart0\_tx queue.
\item \textbf{uart0\_rx\_task} Responsible of checking if any characters in the uart-receive buffer is available. If any, put them into the uart0\_rx queue
\item \textbf{uart1\_tx\_task} Responsibile as uart\_0\_tx\_task
\item \textbf{uart1\_rx\_task} Responsibile as uart\_0\_rx\_task
\item \textbf{can\_rx\_task} Responsible of checking if a CAN-message is available in the MOB. If any put them into the CAN\_rx\_queue
\item \textbf{can\_tx\_task} Responsible of transmitting messages available in the CAN\_tx\_queue
\item \textbf{aq\_node\_task} Responsible of registering the node when AQ boots.
\item \textbf{task\_slip\_decode\_crc\_task} Responsible of running the SLIP \footnote{SLIP described in section \ref{sec:exp8266_firmware}} decapsulation of the received frames \footnote{SLIP described in section \ref{sec:exp8266_firmware}}. The At90CAN128 is not capable of handling the \ac{LLH} as doubles due to a limitation of the compiler,\footnote{The compiler used is AVR-GC++ to compile C++ to the At90CAN128, however it defaults handles a double as 4 bytes and not 8.} so instead the At90CAN128 represents doubles as byte arrays.
When a frame has been received it calculates the CRC of the payload and verifies the CRC bits are the same. The code used for CRC on both PC and At90CAN128 was generated by PyCRC\footnote{\url{https://pycrc.org/ }last visited 29 Maj}.
If CRC is valid, the message is put into \textit{Queue\_gps\_pose} and sent to \textit{aq\_spoof \_task}
\item \textbf{ aq\_spoof\_task} Responsible of converting  received frames from \textit{Queue\_gps\_pose} to CAN messages \footnote{Messages shown in section \ref{sec:system_indoor_drone}}.
\end{itemize}

\subsection*{Test of CAN and Queues}
A test of the CAN and queues was conducted by sending a known ID and data to the extension-board over \ac{CAN}. The At90CAN128 then sent, using UART, to the PC what it received. Then test can be seen in appendix \ref{app:scheduler_and_can_test}.
As expected, the PC received the same ID and data as sent over CAN. This shows CAN and UART is configured correctly and that the queues transfer data between the tasks.
\subsection*{Test of scheduler}
%\input{scheduler_performance_test}
In order to test the timing of the scheduler, a led\_task was written. The task can be seen in code \ref{code:test_scheduler}
\begin{lstlisting}[language = c, caption = RTCS task used in timing test. It sends the state of the task as ASCII character to UART0 and waits 1000 ticks., label=code:test_scheduler]
void is_alive_task(uint8_t my_state){
	UDR0 = my_state+'0'; 	// Write state as ASCII to UART0
	switch(my_state){
	case 0:
		INT_LED_ON_GREEN;  // Other off
	    set_state( 1 );    // Set next state
		break;
	case 1:
		INT_LED_ON_RED;    // Other off
	    set_state( 2 );
		break;
	case 2:
		INT_LED_ON_BLUE;  // Other off
	    set_state( 0 );
		break;
	}
	wait( 1000 ); 	// Wait one second (1000 ticks = 1000 * 1ms )
}
\end{lstlisting}

The test was done by writing the current state of the task to UART0.\\ A python script were made that measures the time between each character received. The script can be seen in code \ref{code:test_rtcs_python}.
\begin{lstlisting}[language = python, caption = Python code used to measure time between received byte, label=code:test_rtcs_python]
# Imports
# Open serial port

t = time.time()
while True:
    char =  ser.read(1):
    print time.time() - t, ","
    t = time.time()

ser.close()
\end{lstlisting}
After receiving 700 bytes the standard deviation and mean was calculated using matlab.
The mean is 1.0089 sec with a standard deviation of 0.0042 sec.

Part of the variance is caused by the inaccuracy of the timing on the PC running the python code. If a more accurate measure was needed, a scope could be attached to the $\mu$C's GPIO. Each time the scheduler enters the task the GPIO should be set high, and when it exists the GPIO should be set low. This was not done due to lack of time.





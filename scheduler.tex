\subsubsection*{Test of RTCS timing}
In order to test the timing of the scheduler, a led\_task was written. The task can be seen in code \ref{code:test_scheduler}
\begin{lstlisting}[language = c, caption = RTCS task used in timing test, label=code:test_scheduler]
void is_alive_task(uint8_t my_state){

	// Write to UART0
	UDR0 = my_state+'0';

	switch(my_state){
	case 0:
		INT_LED_ON_GREEN;
		INT_LED_OFF_RED;
		INT_LED_OFF_BLUE;
	    set_state( 1 );
		break;
	case 1:
		INT_LED_OFF_GREEN;
		INT_LED_ON_RED;
		INT_LED_OFF_BLUE;
	    set_state( 2 );
		break;
	case 2:
		INT_LED_OFF_GREEN;
		INT_LED_OFF_RED;
		INT_LED_ON_BLUE;

		// Set next state
	    set_state( 0 );
		break;
	}
	// Wait one second
	wait( 1000 );
}
\end{lstlisting}

The test was done by writing the current state of the task to UART0.\\ A python script were made that measures the time between each character received. The script can be seen in code \ref{code:test_rtcs_python}.
\begin{lstlisting}[language = python, caption = Python code used to measure time between received byte, label=code:test_rtcs_python]
#!/usr/bin/python
import serial
import time

ser = serial.Serial( port='/dev/ttyUSB0', baudrate=57600 )

t = time.time()
while True:
    for char in ser.read(1):
        print time.time() - t, ","
        t = time.time()

ser.close()
\end{lstlisting}
The output of the script were redirected to a file. After receiving 700 bytes the standard deviation and mean was calculated using matlab.
The mean is 1.0089 sec with a standard deviation of 0.0042 sec.

Part of the variance is caused by the inaccuracy of the timing on the PC running the python code. If a more accurate measure was needed, a scope could be attached to the $\mu$C's GPIO. Each time the scheduler enters the task the GPIO should be set high, and when it exists the GPIO should be set low. The scopes at SDU is capable of telling the variance of the off signal. \\
It can be concluded that the scheduler performs well.
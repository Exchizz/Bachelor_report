Drones are more and more used in different applications in different areas because they make it relatively easy to get a quick or more in depth overview of a situation. A less positive thing of the drones is the amount of energy they are capable of carrying. A drone designed with long time-flight flying in mind is capable of flying approximately 20 minutes depending on weather and payload. If the drone is equipped with a heavy camera or other kind of payload, then the flight time starts to decrease rapidly. So far the solution has been to increase the size of the drones to mount bigger batteries, but this might not be right way of doing it. Drones become less efficient, less responsive and gets more dangerous due to increase in weight.

By looking at the nature, one can easily see how small animals like ants and birds manage to cooperate and thereby build or move bigger things that they would not be able to do on their own. This way of small independent, decentralized units working together is called a swarm.\\

By making drones smaller, they get more efficient, their flight time increase and they get cheaper but of the cost of their ability to lift. Therefore it seems like an idea to make small drones cooperate to solve more complex tasks.\\

SDU is currently using a platform called AutoQuad which is supported by large and small drones. Solving a task as a swarm is complex, and is not in the scope of this project. This project intend to implement a Leader-follower principle inspired by birds flocking behaviour. The leader-drone will have a preprogrammed flight path and the follower-drones will have no knowledge about the flightpath. Each follow-drone will try to keep a distance of 50 cm within plus minus 10 cm to its neighbours and the leader-drone. A computer program will be written to control each drone and tell them where to go without colliding. This will be developed in an indoor controlled environment. Since GPS is unavailable indoor, this project will be using vision, e.g. Henrik Midtiby's MarkerLocator to get the location of each drone. When the Follow-leader has been implemented it should in theory work outside as well, though with larger distances due to GPS inaccuracy.

\subsection{Related Work}
In order to find relevant research about drones flying indoor, a few search phrases was conducted.
The following keywords were used to create different phases: Indoor, environment, swarm, localization, AutoQuad, quadrotor, mini UAV, test facilities, ETH, accurate, RTK, totalstation.
Based on the keywords a few papers was deemed relevant to the project and has been combined in order to give the reader an overview within the field of this project. \\

Developers of AutoQuad did previously try to implement RTKLib \footnote{http://www.rtklib.com} in AQ. Unfortunately they did manage to make it work as expected \footnote{Jussi Hermansen, owner of \url{http://ViaCopter.eu}}.\\

One of the big players within the field of indoor navigation and controlling multiple drones accurately is the university ETHzûrich and their Institute for Dynamic Systems and Modelling \footnote{\url{http://www.idsc.ethz.ch/research-dandrea/research-projects/aerial-construction.html}}. They have developed a test flying area they call "Flying Machine area" which provides facilities for doing prototype testing of new control algorithms \cite{lupashin2014platform}. The FMAs dimensions is 10*10*10m and provides nets to protect people and mattresses to protect the drone  if a crash. The FMA has further been developed into a mobile installation to be used in demonstrations in Europe and North America. One of their demonstrations where used in a TED video about multiroters and their capabilities \footnote{https://www.ted.com/talks/raffaello\_d\_andrea\_the\_astounding\_athletic\_power\_of\_quadcopters}.
The multiroter usually used in the FMA is Ascending Technologies' Hummingbird with custom wireless communication and electronics. \\
They have build it as a module design in order to be easy to replace parts of their system by simulations and to make it scalable.
One of their modules is a copilot that implements an accident handler in case of user-code crashing or sending invalid commands to the drones.
They are using UDP multicast packets as communication between ground computation and flying objects. The use of UDP multicast packets between modules since to makes the system more simple but also to avoid the need of buffers to handle unsuccessful transmissions and retransmissions. \\
In order to detect the quadroters they are using a commercial motion capture system. Three reflective markers is mounted on each flying object in order to obtain attitude and position. They are using three cameras to reduce the risk of false positive even though two cameras would be enough to get a flying objects 6D position.\\
 
 
\cite{kang2015indoor} proposes a more simplistic approaches to do indoor navigation.
They use bluetooth 4 to communicate between their multiroter(Rolling Spider) and an android phone which controls the multiroter.
They have mounted a camera on the ceiling to detect the target and the flying multiroter.
By doing background subtraction they can detect where the drone is in the frame by subtracting the background from each frame \cite{wikiBackgroundsubtraction}. 
By doing a convolution sum, the targets can be located. By analyzing the pixels around the location of the multiroter, they can get the heading. \\

\cite{sanchez2014system} proposes a framework to accelerate the process of prototyping multiroters behaviors. Their framework is designed for a swarm of drones to fly in a environment with obstacles. One of their design requirements is, that the framework should be highly decoupled from the application the researcher is testing in order to speed up the development process. 
Position estimates is obtain by using onboard IMU and optic flow. To avoid expensive motion capture systems they have used markers that can easily be recognized by cameras mounted on the drones to get a absolute 3D estimate. Each obstacle got a ArUco-marker \cite{Aruco2014} that can be detected by the front camera mounted on the multiroter. \\
They have decided to use ROS as middleware to provide generic interfaces between the modules used in their framework. Different multiroters can be used as long they use the same interface. Communication between multiroters and ground station(if used) is done using WIFI. \\


The most common type of indoor localization is using vision where the camera is either mounted in the environment or where the drone is equipped with cameras to obtain position estimates.\\
\cite{stirling2012indoor} uses a different approach where they use a \textit{Robot Sensor Network} to map the environment.
The idea is that each drone can either be a beacon or explorer. Each drone alternates between these two states. Beacons stays still below the ceiling without moving while explorer flies around to unknown locations. Beacons emit IR light in order to triangulate beacons position.  Beacons detecting unexplored locations calls for explorer that will become beacons and so forth until the environment is mapped. To synchronize the beacons 2.4 ghz WIFI is used. When the environment is mapped, graph searching algorithms can be used to find a path through the environment.


\subsection{Problem Statement}
The current AutoQuad version does not support any other source of global positioning than the onboard GPS and thereby does not support accurate flying better than GPS and if GPS is not available, there i is no alternatives supported. For AutoQuad to be used in applications requiring high accuracy or where GPS is unavailable this functionality is needed. \\


\subsection{Hypothesis}
If each drone's 2D position is obtained using vision and spoofed into the drone using CAN, then it is possible for at least 3 drones to follow a leader drone with a preprogrammed flight path and keep a euclidean distance at 50 cm within plus minus 10 cm to the leader and its neighbours.